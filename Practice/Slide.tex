\documentclass{beamer}
\usetheme{JuanLesPins}
\usepackage[sort]{natbib}
\usepackage{tabularx}


% Components of the title page
\title[\LaTeX\ in 24H]{\LaTeX\ in Twenty Four Hours}
\subtitle{A Practical Guide for Scientific Writing}
\author[D. Datta]{Dilip Datta}
\institute[\LaTeX-LT]{\LaTeX\ Learners Team}
\date[L24H :: 21-06-2016]{June 21, 2016}
\titlegraphic{\includegraphics[width=20mm]{logo_LA}}
%
\begin{document}
    % Frame 1
    \frame[plain]{\titlepage}
    % Frame 2
    \section*{Outline}
    \frame[t]{ \frametitle{Presentation outline} \tableofcontents }
    % Frames 3 and 4
    \section[Introduction]{Introduction to \LaTeX}
    \subsection[Definition]{Definition of \LaTeX}
    \frame[t]
    { \frametitle{Introduction to \LaTeX} \framesubtitle{What is \LaTeX?}
    \begin{itemize}
    \item \LaTeX\ is a macro-package for typesetting documents.
    \item \LaTeX\ instructions are interspersed with …
    \item \LaTeX\ input files have .tex extension.
    \item \LaTeX\ output can be obtained in .dvi or .pdf format.
    \end{itemize}
    }
    \subsection[Resources]{Resources on \LaTeX}
    \begin{frame}[t]
    \frametitle{Introduction to \LaTeX} \framesubtitle{Some popular books on \LaTeX}
    \begin{enumerate}
    \item The \LaTeX\ Companion by \citet{Datta-Figueira-2013}
    \item A Guide to \LaTeX2$_\varepsilon$ by \citet{Deb-2001}
    \item \LaTeX: User's Guide and Reference Manual by \citet{Burke-etal-1996}
    \end{enumerate}
    \end{frame}
    % Frame 5
    \section*{ }
    \begin{frame}[t]
    \frametitle{References}
    \bibliographystyle{apalike} 
    \bibliography{mybib2}
    \end{frame}
    % Frame 6
    \section*{ }
    \begin{frame}
    \begin{center}
    \Large{\bf\textcolor{blue}{Thanks a lot}}\\[5mm] … \end{center}
    \end{frame}
%--------------------------------------------    

    \begin{frame}[t]
        \begin{theorem}
        $(a+b)^2 = a^2 + 2ab + b^2$
        \end{theorem}
        %
        \begin{proof}<2->
        $(a+b)^2=(a+b)(a+b)=a^2+2ab+b^2$
        \end{proof}
        %
        \begin{example}<3->[Square of sum]
        $(3+5)^2=3^2+2\times3\times5+5^2=64$
        \end{example}
    \end{frame}

%------------------------------------------

    \begin{frame}[t]
        \begin{block}{Rule}
            The amsmath and amssymb …
        \end{block}
        %
        \begin{alertblock}<2->{Warning}
        A mathematical expression …
        \end{alertblock}
        %
        \begin{exampleblock}<3->{Example}
        $\sin^2\theta + \cos^2\theta = 1$
        \end{exampleblock}
    \end{frame}

%---------------------------------------------    
        \begin{frame}[t]
        \frametitle{Result}
        % Example 1
        \only<1-2> {\color<1-2>[rgb]{1,0.3,0.5}{First year}}
        \only<2> {\color<2>[rgb]{1,0.3,0.5}{
        \begin{table}
        \flushleft
        \begin{tabular}{cccc}
        \hline & {\bf Total} & {\bf Passed} & {\bf Pass rate}\\
        \hline Boys & 56 & 50 & 89.3\%\\
        Girls & 38 & 36 & 94.7\%\\
        \hline
        \end{tabular}
        \end{table} }}
        % Example 2
        \onslide<3-> {\color<3->[rgb]{1,0.3,0.5}{Second year}}
        \onslide<4-> {\color<4->[rgb]{0,0,0}{
        \begin{table}
        \begin{tabularx}{\linewidth}{XXXX}
        \hline & {\bf Total}& {\bf Passed}& {\bf Pass rate}\\
        \hline \uncover<5->{\alert<5>{Boys}}& \uncover<5->{\alert<5>{52}}&
        \uncover<5->{\alert<5>{49}}& \uncover<5->{\alert<5>{94.2\%}}\\
        \uncover<6>{\alert<6>{Girls}}& \uncover<6>{\alert<6>{46}}&
        \uncover<6>{\alert<6>{41}}& \uncover<6>{\alert<6>{89.1\%}}\\
        \hline
        \end{tabularx}
        \end{table} }}
    \end{frame}

\end{document}